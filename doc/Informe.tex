\documentclass[11pt, a4paper]{article}
\usepackage{amsmath} 
\usepackage[spanish]{babel}
\usepackage{graphicx}
\usepackage{blindtext} 
\usepackage{fancyhdr}
\usepackage{float}
\usepackage[colorlinks=true, linkcolor=black, urlcolor=blue]{hyperref}
\usepackage[a4paper,top=2.54cm,bottom=2.54cm,left=3.18cm,right=3.18cm]{geometry}


% Portada del informe
\newgeometry{top=1.5cm, bottom=2.5cm, left=3cm, right=3cm}
\begin{document}
	\thispagestyle{empty}
	\vspace*{0.1cm}
	
	\begin{center}
		\includegraphics[width=0.5\textwidth]{img/logo_unc.png} 
	\end{center}
	\begin{center}
		\Large UNIVERSIDAD NACIONAL DE CÓRDOBA \\
		FACULTAD DE CIENCIAS EXACTAS FÍSICAS Y NATURALES \\
		CÁTEDRA DE ELECTRÓNICA DIGITAL III
		\\[2em]
		TRABAJO PRACTICO INTEGRADOR
		\\[2em]
		\textbf{"SISTEMA DE ADQUISICIÓN, CONVERSIÓN Y PROCESAMIENTO DE UNA SEÑAL DE ELECTROCARDIOGRAMA"}
		\\[2em]
		Grupo Nº 10 \\[1.5em]
		Alumnos:\\[0.25em]
		
		Font, Julián \\[0.25em]
		Krede, Julián \\[0.25em]
		Soria, Francisco \\[2em]
		Profesor:\\[0.25em]
		Migliore, Emiliano Elvio \\[2em]
		Comisión Nº 3\\[1em]
		20 de Noviembre de 2025
	\end{center}
	\restoregeometry
	% Fin portada del informe
	
	\newpage
	%Pagina de indices
	\tableofcontents

	\newpage 
	
	%---------------------------------------------------------------------
	%IMPORTANTE: Descomentar secciones a medida que se vayan implementado
	%---------------------------------------------------------------------
	
	\section{Proyecto}
	El presente Trabajo Práctico Integrador describe la implementación de un Sistema de Adquisición, Conversión y Procesamiento de una señal de Electrocardiograma (ECG), específicamente correspondiente a la Derivación II, destinado al monitoreo en tiempo real de la actividad eléctrica cardíaca.
	
	La señal cardíaca es obtenida del paciente y dirigida al subsistema de adquisición y acondicionamiento de Señal (SSE 2.1) dado que la señal ECG presenta amplitudes del orden de los milivoltios y es altamente susceptible al ruido, esta etapa permite amplificar, filtrar y adecuar la señal antes de su posterior digitalización. Una vez acondicionada, la señal es transferida al sistema microcontrolador LPC1769 (SSEP.2.2), el cuál realiza la conversión analógico-digital y el procesamiento correspondiente para obtener una representación digital estable y útil.
	
	Posteriormente, la señal procesada es remitida al subsistema de comunicación de datos (SSE.2.3) para ser enviada a una computadora, donde puede ser observada en tiempo real. El sistema incorpora un subsistema de señalización acústica (SSE.4.2) destinada a emitir un pulso audible simulando cada latido detectado en la señal ECG y ademas funcionar como alarma técnica en el caso de no detectar pulsaciones. El sistema ademas incorpora un subsistema de señalización visual (SSE.4.1) que se activa en el caso de que se detecta una cantidad de pulsaciones por minuto mayor o menor a valores umbrales establecidos previamente en el sistema microcontrolador.
		
	\begin{figure}[htbp]
		\centering
		\includegraphics[width=1\textwidth]{img/Diagrama-de-bloques.pdf}
		\caption{Diagrama de bloques del sistema}
		\label{fig:diagrama-de-bloques}
	\end{figure}
	
	%\section{Descripción del Sistema Electrónico Programable}
	
	
	
	%\section{Desarrollo del hardware}
	
	%\subsection{Circuito del Subsistema electrónico programable microcontrolador}
	
	%\subsection{Circuito del Subsistema electrónico de señalización óptico y acústico}
	
	%\subsection{Circuito del Subsistema electrónico de visualización de datos}
	
	%\subsection{Circuito del Subsistema electrónico de entrada de datos}
	
	%\subsection{Circuito del Subsistema electrónico de adquisición de señales}
	
	%\subsection{Circuito del Subsistema electrónico de comunicación de datos}
	
	
	
	%\section{Desarrollo del firmware}
	
	%\subsection{Firmware del Subsistema electrónico de señalización óptico y acústico}
	
	%\subsection{Firmware del Subsistema electrónico de visualización de datos}
	
	%\subsection{Firmware del Subsistema electrónico de entrada de datos}
	
	%\subsection{Firmware del Subsistema electrónico de adquisición de señales}
	
	%\subsection{Firmware del Subsistema electrónico de comunicación de datos}
	
	
	
	%\section{Pruebas del sistema}
	
	
	
	%\section{Conclusiones}
	
	
	
	%\section{Bibliografía y referencias}
	
	

\end{document}